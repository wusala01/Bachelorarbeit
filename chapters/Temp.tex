Due to the GUI being the main interface to the underlying functionality, it is exceptionally important to build an easy to use and easy to understand user interface \parencite[cf.][2]{Dray.1995}.


Marketing communication applications have the quirk of having no direct impact of the targeted users on the product. They cannot directly be interviewed for goals and requirements. Additionally the group of targeted users is normaly to large to wrap your head around them. That is the reason for using personas.


\subsection{Minimalist Approach of understandable User Interfaces}

Facing the challenges of Attention Economy,as mentioned in \Cref{ssec:attention} on \cpagerefrange{beginAtt}{endAtt}, business decision makers must explore applications at a higher speed that private users, because as stated above, spend attention implies costs of opportunity \parencite[cf.][]{Bakar.2017}. Therefore, designing a application must be done in optimization for fast knowledge transfer and easy exploration. In order to achieve this goal, \textcite{Bakar.2017} suggests a set of goals:

\begin{enumerate}
\item{Getting Started Fast} implies, that no unnecessary contents are disturbing the user from using the core functionality of the page by drastically removing  \say{explanatory and procedural information and let the user learn [...] through exploration} \parencite{Bakar.2017}.
\item{Training on Real Tasks} leads the user more easily receiving information and being more restrained by the application, due to a sense of excitement.
\item{Reading in Any Order} means the quality of the individual information to be not to have perquisite knowledge of any other page within the application.
\item{Exploiting Prior Knowledge} can help to keep the users attention by mostly presenting new information. This is documented as being difficult to be implemented successfully \sekcite{Carroll.1987}{}{Farkas.1990}{}.
\item{Coordinating System and Training} represents the progress of novice users in learning to work with the software. This is accomplished by keeping the users attention on the user interface and supplying constructive information.
\item{Supporting Error Recognition and Recovery} implies strong testing of the application to determine necessary error recognition and implement sufficient information recovery.
\item{Using the Situation} suggest providing the user many ways to explore the application in means of functionality and informational contents. Offering options for preoccupying oneself with the application amplifies the information transfer but increases the potential for errors.
\item{Developing Optimal Training Design} pertains the correlation of the user interface design to the users needs and behaviours. 
\item{Reasoning and Improvising} relatives the approach of exploratory interaction by offering instructions where compulsory.
\end{enumerate}