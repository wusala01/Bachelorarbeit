\newapp{General Conditions Migration Projects}
\subsection*{Minutes form Memory \rom{1} - Interview with the Sponsor}
\begin{tabular}{l l}
Date: & 20.04.2017 \\
Place: & Franfurt am Main \\
Duration: & 13:00 to 14:00 \\
Expert: & Sachs, Markus \\
Interviewer: & London, Nick \\
Topic: & General condition for migration projects
\end{tabular}

\paragraph{How the projects are acquired:}
The customer decides for whatever reason, that it is necessary, or beneficial to migrate from their current banking software to another. Examples for this could be: a change of the software provider, outsourcing of the banks IT infrastructure, or the fusion of two bank, as well as a joint venture. The customer then sends out a request for proposal, on which companies can apply. After a first level of selection, some candidates are invited for a presentation day. The presentations have certain conditions, such as a set time frame. After the presentation day is over, the bank decides which applicant is selected for the order.
\paragraph{The challenges IBM faces during this procedure:}
IBM is often one of the more expensive providers. Additionally, many perceive the IBM as the hardware company it was in the last decade. They neither see the IBM as a software consulting, nor as a business consulting company.
\paragraph{Does IBM have done successful projects of this kind before?} Yes it has, but those cannot be named in this interview without permission of the customers. From these projects IBM employees have much experience with potential struggles coming up in this setting, plus the in depth knowledge of IBM regarding software and IT in general, IBM is well suited to find technical solutions for most known challenges. And since the purchase of PricewaterhouseCoopers' and lots of other business consulting companies, IBM is prepared to guide the customer in decisions, where technical solutions are not enough.