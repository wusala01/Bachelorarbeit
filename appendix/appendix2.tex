\newapp{Expectations of the Application}
\subsection*{Minutes form Memory \rom{2} - Interview with the Sponsor}
\begin{tabular}{l l}
Date: & 28.04.2017 \\
Place: & Franfurt am Main \\
Duration: & 14:00 to 14:30 \\
Expert: & Sachs, Markus \\
Interviewer: & London, Nick \\
Topic: & Expectations of the application
\end{tabular}

\paragraph{What is the goal of the application?} The goal for the application is to have a application we can give the customer after a proposal presentation, so he can play with it. 

\paragraph{How do you imagine the application?} We already did some research on the look banking applications have. We want the application to  look like a stereotypical example of a banking app, in which the user can click around and find hints on how mistakes in the project may lead to negative influences on the customer. But not only negative points, also things to consider, ans some basic information to show that we know the business.

\paragraph{In which situation should the customer use the application?} The customer should use it at his way back home in the train, or somewhere he has some spare time. Just go on there and get some quick information. 

\paragraph{Is this going to be an intallable application?} We do not know yet. We neither know how the process is to get an app on the stores for IBM and we also do not know what real alternatives we have. You see, we want the customer to use it on the way, and we all know that you can loose connection there some time.

\paragraph{So are you suggesting an offline capability?} Yes the offline capability is one aspect, but not key. First the professional questions, then the technical one.

\paragraph{Will you let some developer in India program the app for you?} No we want it students to do. That is cheaper.

\paragraph{You know they are limited in time right?} Yes, that is why the requirements must be packet into chunks, that one can do one after the other.

